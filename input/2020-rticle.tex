% !TeX root = RJwrapper.tex
\title{Validating Neural Network Packages in R with \pkg{NNbenchmark} --
working title}
\author{by Salsabila Mahdi, Akshaj Verma, John C. Nash, Christophe
Dutang, Patrice Kiener}

\maketitle

\abstract{%
An abstract of less than 150 words.
}

\hypertarget{abstract}{%
\subsection{Abstract}\label{abstract}}

\hypertarget{introduction}{%
\subsection{Introduction}\label{introduction}}

R Statistical Software, as any opensource platform, has relied on its
contributors to keep it up to date with the latest developments. One of
the many is neural networks. Neural networks are a class of models,
based on the brain's own connections system, in the growing field of
machine learning. Before, they were a theory with not much practical
implementation, partly because of how computationally demanding their
algorithms are. A neural network algorithm involves complicated
calculations for each gradual ``step'' into the direction of the optimal
solution, namely, determining the gradient of the cost function
recursively. Parameters are adjusted accordingly for each iteration from
this information.\\
Manually computing partial gradients from complex nonlinear models is
taxing. With the help of modern technology and its high-level code,
implementing a neural network is relatively simple.

However, the utilization of computers should not only be done
effectively but also efficiently. Instead of calculating first-order
derivatives, and moving incrementally forward by a predetermined
learning rate, it is better to adjust the size of each step according to
its curvature. The second-order derivative provides this information.
Numerical methods can take this further. The Hessian, a matrix of the
second derivatives, is approximated instead of perfectly derived. Such
algorithms are known as Quasi-Newton algorithms.
Broyden-Fletcher-Goldfarb-Shanno (BFGS) is a well-known example of an
algorithm from this class. We believed that these second-order
algorithms would perform better than first-order algorithms in terms of
finding the optimal solution.

Regardless of our belief, it is crucial to conduct a thorough
examination to assess the quality of these training algorithms in R.
There is much code, but barely any comparison. In particular, packages
that provide neural network of the perceptron type (one input layer, one
normalized layer, one hidden layer with a nonlinear activation function
that is usually tanh(), one normalized layer, one output output layer)
for regression purpose (i.e.~NN(X1, ., Xn) = E{[}Y{]} were the focus of
this research. At the very least, this research will serve as a
framework for future research on neural network benchmarking.

\hypertarget{methodology}{%
\section{Methodology}\label{methodology}}

Our research process was largely divided into teh following 3 phases.

\hypertarget{phase-1---preparation}{%
\subsection{Phase 1 - Preparation}\label{phase-1---preparation}}

\hypertarget{datasets-need-to-be-finished}{%
\subsubsection{Datasets =\textgreater{} NEED TO BE
FINISHED}\label{datasets-need-to-be-finished}}

All the datasets used are nonlinear. Linear data sets are more simple
and can even be solved with OLS (Ordinary Least Squares) regression.
This is why we believe to truly set apart the ability of neural networks
we needed to go beyond linear regression. Varying difficulties between
data sets helped to classify further package's algorithms accuracy. One
site was used for 3 of the multivariate data sets. Sonja Surjanovic and
Derek Bingham of Simon Fraser University created this resourceful
website to evaluate the design and analysis of computer models. Links to
each dataset and their level of difficulty:\\
- \url{http://www.sfu.ca/~ssurjano/fried.html} (Friedman - average)\\
- \url{http://www.sfu.ca/~ssurjano/detpep10curv.html} (Dette - medium)\\
- \url{http://www.sfu.ca/~ssurjano/ishigami.html} (Ishigami - high)\\
The other multivariate dataset, Ref153, was taken from \ldots{} 3 of the
univariate datasets were taken from NIST at:
\url{https://www.itl.nist.gov/div898/strd/nls/nls_main.shtml}. Gauss1
and Gauss2 have a low level of difficulty to solve. Gauss3 is average.
Dmod1, Dmod2 are from \ldots{} Dreyfus1 is a pure neural network which
has no error. This can make it difficult for algorithms that assume an
error exists. Dreyfus2 is Dreyfus1 with errors. NeuroOne from \ldots{}\\
Wood \ldots{}

\hypertarget{packages}{%
\subsubsection{Packages}\label{packages}}

Searching through the thousands of packages title or the package
description one by one would have taken a long time. With
\CRANpkg{RWsearch} \citep{R-RWsearch} we were able to automate the
process. All packages that have ``neural network'' as a keyword in the
package title or in the package description were included.

(to use when needed in text, testing how long references will be) 1
\CRANpkg{AMORE} \citep{R-AMORE}, 2 \CRANpkg{ANN2} \citep{R-ANN2}, 3
\CRANpkg{appnn} \citep{R-appnn}, 4 \CRANpkg{autoencoder}
\citep{R-autoencoder}, 5 \CRANpkg{automl} \citep{R-automl}, 6
\CRANpkg{BNN} \citep{R-BNN}, 7 \CRANpkg{brnn} \citep{R-brnn}, 8
\CRANpkg{Buddle} \citep{R-Buddle}, 9 \CRANpkg{CaDENCE}
\citep{R-CaDENCE}, 10\CRANpkg{cld2} \citep{R-cld2}, 11\CRANpkg{cld3}
\citep{R-cld3}, 12\CRANpkg{condmixt} \citep{R-condmixt},
13\CRANpkg{DamiaNN} \citep{R-DamiaNN}, 14\CRANpkg{deep} \citep{R-deep},
15\CRANpkg{deepdive} \citep{R-deepdive}, 16\CRANpkg{deepnet}
\citep{R-deepnet}, 17\CRANpkg{deepNN} \citep{R-deepNN}, 18\CRANpkg{DNMF}
\citep{R-DNMF}, 19\CRANpkg{elmNNRcpp} \citep{R-elmNNRcpp},
20\CRANpkg{ELMR} \citep{R-ELMR}, 21\CRANpkg{EnsembleBase}
\citep{R-EnsembleBase}, 22\CRANpkg{evclass} \citep{R-evclass},
23\CRANpkg{gamlss.add} \citep{R-gamlss.add}, 24\CRANpkg{gcForest}
\citep{R-gcForest}, 25\CRANpkg{GMDH} \citep{R-GMDH}, 26\CRANpkg{GMDH2}
\citep{R-GMDH2}, 27\CRANpkg{GMDHreg} \citep{R-GMDHreg}, 28\CRANpkg{gnn}
\citep{R-gnn}, 29\CRANpkg{grnn} \citep{R-grnn}, 30\CRANpkg{h2o}
\citep{R-h2o}, 31\CRANpkg{hybridEnsemble} \citep{R-hybridEnsemble},
32\CRANpkg{isingLenzMC} \citep{R-isingLenzMC}, 33\CRANpkg{keras}
\citep{R-keras}, 34\CRANpkg{kerasR} \citep{R-kerasR}, 35\CRANpkg{leabRa}
\citep{R-leabRa}, 36\CRANpkg{learNN} \citep{R-learNN},
37\CRANpkg{LilRhino} \citep{R-LilRhino}, 38\CRANpkg{minpack.lm}
\citep{R-minpack.lm}, 39\CRANpkg{MachineShop} \citep{R-MachineShop},
40\CRANpkg{monmlp} \citep{R-monmlp}, 41\CRANpkg{neural}
\citep{R-neural}, 42\CRANpkg{neuralnet} \citep{R-neuralnet},
43\CRANpkg{NeuralNetTools} \citep{R-NeuralNetTools},
44\CRANpkg{NeuralSens} \citep{R-NeuralSens}, 45\CRANpkg{NlinTS}
\citep{R-NlinTS}, 46\CRANpkg{nlsr} \citep{R-nlsr}, 47\CRANpkg{nnet}
\citep{R-nnet}, 48\CRANpkg{nnetpredint} \citep{R-nnetpredint},
49\CRANpkg{nnfor} \citep{R-nnfor}, 50\CRANpkg{nntrf} \citep{R-nntrf},
51\CRANpkg{nnli2bRcpp} \citep{R-nnlib2Rcpp}, 52\CRANpkg{onnx}
\citep{R-onnx}, 53\CRANpkg{OptimClassifier} \citep{R-OptimClassifier},
52\CRANpkg{OSTSC} \citep{R-OSTSC}, 53\CRANpkg{pnn} \citep{R-pnn},
54\CRANpkg{polyreg} \citep{R-polyreg}, 55\CRANpkg{predictoR}
\citep{R-predictoR}, 56\CRANpkg{qrnn} \citep{R-qrnn},
57\CRANpkg{QuantumOps} \citep{R-QuantumOps}, 58\CRANpkg{quarrint}
\citep{R-quarrint}, 59\CRANpkg{radiant.model} \citep{R-radiant.model},
60\CRANpkg{rasclass} \citep{R-rasclass}, 61\CRANpkg{rcane}
\citep{R-rcane}, 62\CRANpkg{regressoR} \citep{R-regressoR},
63\CRANpkg{rminer} \citep{R-rminer}, 64\CRANpkg{rnn} \citep{R-rnn},
65\CRANpkg{RSNNS} \citep{R-RSNNS}, 66\CRANpkg{ruta} \citep{R-ruta},
67\CRANpkg{simpleNeural} \citep{R-simpleNeural}, 68\CRANpkg{snnR}
\citep{R-snnR}, 69\CRANpkg{softmaxreg} \citep{R-softmaxreg},
70\CRANpkg{Sojourn.Data} \citep{R-Sojourn.Data}, 71\CRANpkg{spnn}
\citep{R-spnn}, 72\CRANpkg{TeachNet} \citep{R-TeachNet},
73\CRANpkg{tensorflow} \citep{R-tensorflow}, 74\CRANpkg{tfestimators}
\citep{R-tfestimators}, 75\CRANpkg{trackdem} \citep{R-trackdem},
76\CRANpkg{TrafficBDE} \citep{R-TrafficBDE}, 77\CRANpkg{tsensembler}
\citep{R-tsensembler}, 78\CRANpkg{validann} \citep{R-validann},
80\CRANpkg{zFactor} \citep{R-zFactor}.

\hypertarget{phase-2---exploration-of-each-package-and-development-of-template}{%
\subsection{Phase 2 - Exploration of each package and development of
template}\label{phase-2---exploration-of-each-package-and-development-of-template}}

\textbf{Exploration}\\
However, not all packages that had the keyword were fit for the scope of
our research. Some didn't have any functions to make neural networks.
They were simply meta-packages. Others were not regression neural
networks of the perceptron type or were only made for specific purposes.
We learned this through reading documentation and trying out example
code. \textbf{Template =\textgreater{} TO REVISE AFTER 2020 CODE}\\
As we inspected the packages, we developed a template for benchmarking.
This template's structure is as follows:\\
(1) Set up of environment - loading packages, setting directory,
options;\\
(2) Summary of datasets;\\
(3) A loop over datasets which contained (a) setting parameters for a
specific dataset (b) selecting benchmark options (c) the training of a
neural network with a package's tuned functions (d) calculation of RMSE
and MAE (e) plot each training over one initial graph, then plot the
best result (f) adding results to the appropriate *.csv file and (g)
clearing up environment for next loop; and\\
(4) Clearing up the environment for the next package. (5) It is optional
to print warnings.\\
This process was made easier with tools from the NNbenchmark package.
(FROM LAST YEAR, HOPEFULLY WE CAN SAY OTHERWISE THIS YEAR:) It is not on
CRAN yet and can instead be found at
\url{https://github.com/pkR-pkR/NNbenchmark}. Our templates for each
package can be found in the companion repository,
\url{https://github.com/pkR-pkR/NNbenchmarkTemplates}.

\hypertarget{phase-3---collection-of-and-analysis-of-results}{%
\subsection{Phase 3 - Collection of and analysis of
results}\label{phase-3---collection-of-and-analysis-of-results}}

\textbf{Collection} After the templates were finished, the packages were
looped on all datasets. Results were collected in the directory of the
templates repository. \textbf{Analysis} To rank the how well a package
converged and its speed, we developed the following method: 1. The
results datasets are loaded into the R environment as one large list.
The dataset names, package:algorithm names and all 10 run numbers,
durations, and RMSE are extracted from that list 2. For the duration
score (DUR), the duration is averaged by dataset. 3 criteria for the
RMSE score by dataset are calculated: a. The minimum value of RMSE for
each package:algorithm as a measure of their best performance b. The
median value of RMSE for each package:algorithm as a measure of their
average performance, without the influence of outliers c.~The spread of
the RMSE values for each package which is measured by the difference
between the median and the minimum RMSE (d51) 3. Then, the ranks are
calculated for every dataset and the results are merged into one wide
dataframe. a. The duration rank only depends on the duration. b. For
minimum RMSE values, ties are decided by duration mean, then the RMSE
median c.~For median RMSE values, ties are decided by the RMSE minimum,
then the duration mean d.~The d51 rank only depends on itself 4. A
global score for all datasets is found by a sum of the ranks (of
duration, minimum RMSE, median RMSE, d51 RMSE) of each package:algorithm
for each dataset 5. The final table is the result of ranking by the
global minimum RMSE scores for each package:algorithm

To rank how easy or not a package was to use (TO BE DISCUSSED FURTHER):
- Functionality (util): scaling, input, output, trace - Documentation
(docs): examples, structure/functions, vignettes

\hypertarget{results}{%
\section{Results}\label{results}}

\textbf{Tables} (NOTE: FINAL MEASURE FOR CONVERGENCE - RMSE RANKS? OR A
COMBINATION OF OTHER MEASURES? As in Christophe's recent email: L1
MAE(), L2 RMSE(), Linfinity (WAE))

(ALSO: THE FOLLOWING IS SIMPLY ALPHABETIC LIST FOR ALL TESTED, I WILL
DIVIDE THE TABLE INTO 4: 2nd ORDER always recommended, 1st ORDER
recommended, 1st ORDER not recommended, untested packages)

\begin{center}
\textbf{Table X: Ratings}
\begin{tabular}{l l l l l l l}
  \toprule
  No & Name (package::algorithm)        & RMSE & DUR & UTIL & DOCS & OVERALL \\
  \midrule
  1  &\pkg{AMORE}::train.ADAPTgd        &      &     &      &      &         \\
     &\pkg{AMORE}::train.ADAPTgdwm      &      &     &      &      &         \\
     &\pkg{AMORE}::train.BATCHgd        &      &     &      &      &         \\ 
     &\pkg{AMORE}::train.BATCHgdwm      &      &     &      &      &         \\
  2  &\pkg{automl}                      &      &     &      &      &         \\
  3  &\pkg{ANN2}::neuralnetwork.sgd     &      &     &      &      &         \\
     &\pkg{ANN2}::neuralnetwork.adam    &      &     &      &      &         \\
     &\pkg{ANN2}::neuralnetwork.rmsprop &      &     &      &      &         \\
  4  &\pkg{brnn}                        &      &     &      &      &         \\
  5  &\pkg{CaDENCE}                     &      &     &      &      &         \\
  6  &\pkg{deepnet}::gradientdescent    &      &     &      &      &         \\
  7  &\pkg{elmNNRcpp}                   &      &     &      &      &         \\
  8  &\pkg{ELMR}                        &      &     &      &      &         \\
  9  &\pkg{h2o}::deeplearning           &      &     &      &      &         \\
  10 &\pkg{keras}                       &      &     &      &      &         \\
  11 &\pkg{kerasformula}                &      &     &      &      &         \\
  12 &\pkg{kerasR}                      &      &     &      &      &         \\
  13 &\pkg{minpack.lm}::nlsLM           &      &     &      &      &         \\
  14 &\pkg{MachineShop}::fit.NNetModel()&      &     &      &      &         \\
  15 &\pkg{monmlp}::fit.BFGS            &      &     &      &      &         \\
     &\pkg{monmlp}::fit.Nelder-Mead     &      &     &      &      &         \\
  16 &\pkg{neural}::mlptrain            &      &     &      &      &         \\
  17 &\pkg{neuralnet}::backprop         &      &     &      &      &         \\
     &\pkg{neuralnet}::rprop+           &      &     &      &      &         \\
     &\pkg{neuralnet}::rprop-           &      &     &      &      &         \\
     &\pkg{neuralnet}::sag              &      &     &      &      &         \\
     &\pkg{neuralnet}::slr              &      &     &      &      &         \\
  18 &\pkg{nlsr}::nlxb                  &      &     &      &      &         \\
  19 &\pkg{nnet}::nnet.BFGS             &      &     &      &      &         \\
  20 &\pkg{qrnn}::qrnn.fit              &      &     &      &      &         \\
  21 &\pkg{radiant.model}::radiant.model&      &     &      &      &         \\
  22 &\pkg{rcane}::rlm                  &      &     &      &      &         \\
  23 &\pkg{rminer}::fit                 &      &     &      &      &         \\
  24 &\pkg{RSNNS}::BackpropBatch        &      &     &      &      &         \\
     &\pkg{RSNNS}::BackpropChunk        &      &     &      &      &         \\
     &\pkg{RSNNS}::BackpropMomentum     &      &     &      &      &         \\
     &\pkg{RSNNS}::BackpropWeightDecay  &      &     &      &      &         \\
     &\pkg{RSNNS}::Quickprop            &      &     &      &      &         \\
     &\pkg{RSNNS}::Rprop                &      &     &      &      &         \\
     &\pkg{RSNNS}::SCG                  &      &     &      &      &         \\
     &\pkg{RSNNS}::Std-Backpropagation  &      &     &      &      &         \\
  25 &\pkg{ruta}                        &      &     &      &      &         \\
  26 &\pkg{simpleNeural}::sN.MLPtrain   &      &     &      &      &         \\ 
  27 &\pkg{snnR}                        &      &     &      &      &         \\
  28 &\pkg{softmaxreg}                  &      &     &      &      &         \\
  29 &\pkg{tensorflow}::AdadeltaOptmizer&      &     &      &      &         \\
     &\pkg{tensorflow}::AdagradOptmizer &      &     &      &      &         \\
     &\pkg{tensorflow}::AdamOptmizer    &      &     &      &      &         \\
     &\pkg{tensorflow}::FtrlOptmizer    &      &     &      &      &         \\
     &\pkg{tensorflow}::GradientDescent &      &     &      &      &         \\
     &\pkg{tensorflow}::MomentumOptmizer&      &     &      &      &         \\
  30 &\pkg{tfestimators}                &      &     &      &      &         \\
  31 &\pkg{tsensembler}                 &      &     &      &      &         \\
  32 &\pkg{validann}::Nelder-Mead       &      &     &      &      &         \\
     &\pkg{validann}::BFGS              &      &     &      &      &         \\
     &\pkg{validann}::CG                &      &     &      &      &         \\
     &\pkg{validann}::L-BFGS-B          &      &     &      &      &         \\
     &\pkg{validann}::SANN              &      &     &      &      &         \\  
     &\pkg{validann}::Brent             &      &     &      &      &         \\
  \end{tabular}
\end{center}

(THE FOLLOWING IS JUST AN ALPHABETICALLY ORDERED LIST OF CURRENTLY
UNTESTED PACKAGES)

\begin{center}
\textbf{Table 2: Review of Ommitted Packages}

\begin{tabular}{l l l l}
  \toprule
  No & Name (package)            & Category & Comment \\
  \midrule
  1  &\pkg{appnn}                & -        & \\
  2  &\pkg{autoencoder}          & -        & \\     
  3  &\pkg{BNN}                  & -        & \\
  4  &\pkg{Buddle}               & -        & \\
  5  &\pkg{cld2}                 & -        & \\
  6  &\pkg{cld3}                 & -        & \\
  7  &\pkg{condmixt}             & -        & \\
  8  &\pkg{DALEX2}               & -        & \\
  9  &\pkg{DamiaNN}              & -        & \\
  10 &\pkg{DChaos}               & -        & \\
  11 &\pkg{deepNN}               & -        & \\
  12 &\pkg{DNMF}                 & -        & \\
  13 &\pkg{EnsembleBase}         & -        & \\
  14 &\pkg{evclass}              & -        & \\
  15 &\pkg{gamlss.add}           & -        & \\
  16 &\pkg{gcForest}             & -        & \\
  17 &\pkg{GMDH}                 & -        & \\
  18 &\pkg{GMDH2}                & -        & \\
  19 &\pkg{GMDHreg}              & -        & \\
  20 &\pkg{grnn}                 & -        & \\
  21 &\pkg{hybridEnsemble}       & -        & \\ 
  22 &\pkg{isingLenzMC}          & -        & \\
  23 &\pkg{leabRa}               & -        & \\      
  24 &\pkg{learNN}               & -        & \\     
  25 &\pkg{LilRhino}             & -        & \\
  26 &\pkg{NeuralNetTools}       & -        & tools for neural networks           \\
  27 &\pkg{NeuralSens}           & -        & tools for neural networks           \\
  28 &\pkg{NlinTS}               & NA       & Time Series                         \\
  29 &\pkg{nnetpredint}          & -        & confidence intervals for NN          \\
  30 &\pkg{nnfor}                & NA       & Times Series, uses neuralnet         \\
  31 &\pkg{onnx}                 & -        & provides an open source format       \\
  32 &\pkg{OptimClassifier}      & NA       & choose classifier parameters, nnet   \\
  33 &\pkg{OSTSC}                & -        & solving oversampling classification  \\
  34 &\pkg{pnn}                  & NA       & Probabilistic                        \\
  35 &\pkg{polyreg}              & -        & polyregression ALT to NN             \\
  36 &\pkg{predictoR}            & NA       & shiny interface, neuralnet           \\
  37 &\pkg{QuantumOps}           & NA       & classifies MNIST, Schuld (2018)      \\
  38 &\pkg{quarrint}             & NA       & specified classifier for quarry data \\
  39 &\pkg{rasclass}             & NA       & classifier for raster images, nnet?  \\
  40 &\pkg{regressoR}            & NA       & a manual rich version of predictoR   \\
  41 &\pkg{rnn}                  & NA       & Recurrent                            \\
  42 &\pkg{Sojourn.Data}         & NA       & sojourn Accelerometer methods, nnet? \\
  43 &\pkg{spnn}                 & NA       & classifier, probabilistic            \\
  44 &\pkg{TeachNet}             & NA       & classifier, selfbuilt, slow          \\
  45 &\pkg{trackdem}             & NA       & classifier for particle tracking     \\
  46 &\pkg{TrafficBDE}           & NA       & specific reg, predicting traffic     \\
  47 &\pkg{zFactor}              & NA       & 'compressibility' of hydrocarbon gas \\
\end{tabular}

\end{center}

\hypertarget{discussion-and-recommandations}{%
\subsection{Discussion and
Recommandations}\label{discussion-and-recommandations}}

A. Recommended: 2nd order algorithms Out of all the algorithms, these
second algorithms generally performed better in terms of convergence
despite being set to a much lower number of iterations, 200, than the
first-order algorithms. Moreover, they performed better in terms of
speed. The best in this class were \CRANpkg{minpack.lm} and
\CRANpkg{nlsr}, tied at rank number 1. The Levenberg-Marquardt (LM)
algorithm used is fast and converges well. stats::nls() is used.
However, these packages require a handwritten formula that may not be
ideal for certain situations. A more popular package for neural networks
is nnet. This might be because it is part of base R. It implements the
BFGS algorithm with stats::optim().

Ranked directly after are some packages that depend on nnet or use the
same functions. They differ in how well they decide initial parameters.
rminer (rank 4), MachineShop (rank 5), and radiant.model (rank 7) use
nnet. Note, radiant.model has its iterations set to 10000, which
originally made it slower yet converge better. We used a modified
version of the package. At rank 6 is validann's BFGS algorithm using
stats::optim(). Its use of optim's L-BFGS-B ranked at number 9 with
CaDENCE's use of optim's BFGS. \CRANpkg{monmlp}, from the same author as
CaDENCE (Alex Cannon), uses the package \CRANpkg{optimx}'s BFGS
\citep{R-optimx}.

Alex Cannon also implemented a quantile regression neural network in
qrnn with stats::nlm(). It requires more iterations and is not as fast
compared to the other second-order algorithms. However, it is a valuable
implementation of quantile regression. Last but not least is
\CRANpkg{brnn}'s Gauss Newton algorithm which ranks at number 8. brnn is
easy to use but does not converge as well due to a hidden constraint: a
missing first parameter. Furthermore, brnn's algorithm minimizes the sum
of squared errors and a penalty on parameters instead of just the sum of
squared errors. This may prevent parameters to get highly correlated,
especially with an almost degenerated Jacobian matrix.

B. Recommended: 1st order algorithms validann optim CG RSNNS SCG h2o
back-propagation RSNNS Rprop ANN2 adam CaDENCE Rprop -SLOW deepnet BP
AMORE ADAPTgdwm AMORE ADAPTgd ANN2 sgd automl trainwgrad ANN2 rmsprop
RSNNS BackpropChunk RSNNS BackWeightDecay RSNNS Std\_Backpropagation
RSNNS BackpropMomentum automl trainwpso validann optim NelderMead snnR
Semi Smooth Newton RSNNS BackpropBatch validann optim SANN monmlp optimx
Nelder Mead

C. Not recommended: 1st order algorithms \textless- DISCUSS CUTOFF By
package ELMR, elmNNRcpp - fast ELM algorithms. Unfortunately, can't
finetune, does not converge well. neuralnet: a large ammount of
iterations, slow, erratic failures tensorflow: NOT EASY TO USE,
subsequently keras, tfestimators, ruta \ldots{} user needs to understand
the language However, advanced users might be able to highly specify a
neural network to their needs (customization?)

By algorithm: neuralnet rprop+ neuralnet rprop- neuralnet slr - once
ranked well with 100000 iterations AMORE BATCHgd CaDENCE pso psoptim -
need to reconfigure? elmNNRcpp - fast, no iterations RSNNS Quickprop (?)
AMORE BATCHgdwm tensorflow MomentumOptimizer tensorflow AdamOptimizer
ELMR - fast, no iterations tensorflow GradientDescentOptimizer keras
rmsprop keras adagrad keras sgd keras adadelta tensorflow
AdagradOptimizer keras adam tensorflow FtrlOptimizer neuralnetwork sag
tensorflow AdadeltaOptimizer neuralnet backprop - note, might not
actually reflect standings, somehow from template to template the
learning rate disappeared. Will fix this in future runs

D. Untested =\textgreater{} TO DO - LIST

\hypertarget{conclusion-to-do-after-2020-code}{%
\section{Conclusion =\textgreater{} TO DO AFTER 2020
CODE}\label{conclusion-to-do-after-2020-code}}

\hypertarget{future-work}{%
\subsection{Future work}\label{future-work}}

As the alogrithms for neural networks continue to grow, there will
always be more to validate. For current algorithms in R, our research
should be extended to encompass more types of neural networks and their
data formats (classifier neural networks, recurrent neural networks, and
so on). Different rating schemes and different parameters for package
functions can also be tried out.

\hypertarget{acknowledgements}{%
\subsection{Acknowledgements}\label{acknowledgements}}

This work was possible due to the support of the Google Summer of Code
initiative for R during years 2019 and 2020. Students Salsabila Madhi
(2019 and 2020) and Akshaj Verma (2019) are grateful to Google for the
financial support.

\bibliography{RJreferences}

\begin{itemize}
\tightlist
\item
  The dreamed NN package: Recommandation to package authors
\item
  Conclusion
\item
  Acknowledgments
\end{itemize}

For the acknowledgements, maybe : « » + later some aknowledgements to
the referees.

How do we proceed?


\address{%
Salsabila Mahdi\\
Affiliation\\
line 1\\ line 2\\
}
\href{mailto:author1@work}{\nolinkurl{author1@work}}

\address{%
Akshaj Verma\\
Affiliation\\
line 1\\ line 2\\
}
\href{mailto:akshajverma7@gmail.com}{\nolinkurl{akshajverma7@gmail.com}}

\address{%
John C. Nash\\
Affiliation\\
line 1\\ line 2\\
}
\href{mailto:jcnash@gmail.com}{\nolinkurl{jcnash@gmail.com}}

\address{%
Christophe Dutang\\
University of Paris-Dauphine\\
line 1\\ line 2\\
}
\href{mailto:dutang@ceremade.dauphine.fr}{\nolinkurl{dutang@ceremade.dauphine.fr}}

\address{%
Patrice Kiener\\
InModelia\\
5 rue Malebranche, 75005 Paris, France\\
}
\href{mailto:patrice.kiener@inmodelia.com}{\nolinkurl{patrice.kiener@inmodelia.com}}
